\chapter{Conclusions and Future Works}
\label{chapter:6_conclusions_future_developments}
In this concluding chapter, I present some final thoughts and propose potential directions for future research concerning the development of PEO.


\section{Conclusions}
\label{section:6_1_conclusions}
Inspired by the curiosity to explore an emerging field like prompt engineering and driven by the need to create a comprehensive resource that could be useful to those who share the same interest, we created PEO. During the design and encoding phases, we applied more modern techniques of ontology engineering. This allowed us to make PEO an ontology capable of fully describing all concepts related to prompt engineering and LLMs, considering the interconnections between them. Special attention was given to the experimentation and evaluation phase to ensure the quality of the ontology as well as its consistency with the specifications. The publication of all artifacts online, according to the FAIR principles, makes PEO visible and editable by any user under the Creative Commons license. Users can further advance the project through various future developments. 

\section{Future Works}
\label{section:6_2_future}
The future developments of the PEO ontology are numerous and can proceed in different yet complementary directions. 
Among these, we consider as interesting the following future developments:
\begin{itemize}
    \item Automatic population using LLMs: experiment with other LLMs and applying more advanced techniques for ontology population.
   
    \item Integration with external ontologies: integrate with the ontologies on LLMs 

    \item Improvement of the inference part: further enhance the expressivity of the TBox to gain even more inference capabilities.


    \item PEO-based AI agent: the development of a PEO-based AI agent can dynamically select and refine prompts based on user intent, task type, and LLM capabilities, optimizing interactions with language models. 
    
 
    \item PEO ontology web application: the development of a web application using HTML, CSS, and JavaScript to interact with the PEO ontology is a potential future advancement that could enhance its accessibility and support its wider adoption. 
\end{itemize}
The future developments that will be carried out will surely enhance PEO, a unique project never experimented that was casually born during a Teams call with my advisor on a warm day in mid-May 2024.
